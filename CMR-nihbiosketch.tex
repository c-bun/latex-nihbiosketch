%!TEX TS-program = xelatex
\documentclass{nihbiosketch}

%%%%%%%%%%%%%%%%%%%%% c-bun's additions
\newcommand{\comment}[1]{\textit{\textcolor{red}{#1}}}
%%%%%%%%%%%%%%%%%%%%% end c-bun's additions

% \usepackage{draftwatermark}  % delete this in your document!
% \SetWatermarkText{Sample}    % delete this in your document!
% \SetWatermarkLightness{0.9}  % delete this in your document!

%------------------------------------------------------------------------------

\name{Rathbun, Colin Michael}
\eracommons{colinrathbun}
\position{Postdoctoral Researcher}

\begin{document}
%------------------------------------------------------------------------------

\begin{education}
Hope College, Holland MI  & B.S & 05/2012 & Chemistry \\
University of California, Irvine  & Ph.D. & 05/2018 & Organic Chemistry \\
University of Colorado, Boulder & Postdoctoral  & 05/2020 & Chemical Biology and Fluorescence Imaging \\
\end{education}


\section{Personal Statement}

\begin{statement}
Methods and tool development have characterized my academic career and continue
to inspire my future work. Throughout my undergraduate and graduate work I
enjoyed designing solutions to problems in both organic chemistry and chemical
biology. My undergraduate research involved development and study of new
transition-metal catalyzed reactions involving carbon--carbon bond activation.
Under the mentorship of professor Jeffrey Johnson, this work lead to some of the
first kinetic characterizations of C--C bond activation reactions. During the
summer of 2011 I had the opportunity to conduct research in Buenos Aires,
Argentina through an NSF REU grant. In graduate school, I continued similar work
in organometallic catalysis during my first two years in the lab of Professor Vy
Dong. My work in the Dong Lab sought efficient transformations for the synthesis
of carbohydrates; a longstanding problem in organic chemistry. Following
advancement, I moved to the Prescher lab
to pursue the development of new tools for bioluminescence imaging. In the
subsequent three years, I developed a new screening platform and image analysis
techniques. The breadth and depth of my previous work has
equipped me to excel in the Palmer lab, where I plan to use my skills in
organic chemistry, chemical biology, and data science to tackle new problems in
RNA imaging. Thus far I have developed a modular synthesis for a new class of
fluorescent RNA imaging probes. I plan to evaluate these molecules in parallel with
libraries of riboswitch binding partners to track individual RNAs as they move throughout living cells.
\comment{Replace with discussion of postdoc and plans for independent lab. Add the biorxiv paper to the top below? Amy has four papers, so that should still be okay? Looks like she has a review on her list as well? maybe put that at the top?}

\begin{enumerate}

	\item \comment{RNA review here??}

  \item \underline{Rathbun, C. M.}*; Porterfield, W. B.*; Jones, K. A.*; Sagoe, M. J.; Reyes, M. R.; Hua, C. T.; Prescher, J. A. ``Parallel screening for rapid identification of orthogonal bioluminescent tools." \textit{ACS Cent. Sci.}, \textbf{2017}, \textit{3}, 1254.

  \item Chen, I. H.; Kou, K. G. M.; Le, D. N.; \underline{Rathbun, C. M.}; Dong, V. M. ``Recognition and Site-Selective Transformation of Monosaccharides by Using Copper(II) Catalysis." \textit{Chem. Eur. J.}, \textbf{2014}, \textit{20}, 5013.

  \item \underline{Rathbun, C. M.}; Johnson, J. B. ``Rhodium-Catalyzed Acylation of Quinolinyl Ketones: Carbon-Carbon Single Bond Activation as the Turnover Limiting Step of Catalysis." \textit{J. Am. Chem. Soc.}, \textbf{2011}, \textit{133}, 2031.

\end{enumerate}

\end{statement}

%------------------------------------------------------------------------------
\section{Positions and Honors}

\subsection*{Positions and Employment}
\begin{datetbl}
2010 -- 2012  & Undergraduate Researcher, Department of Chemistry, Hope College, Holland, MI \\
2010          & Undergraduate Researcher, University of Buenos Aires, Buenos Aires, Argentina \\
2012 -- 2018  & Graduate Researcher, Department of Chemistry, University of California, Irvine \\
2018 -- 2020  & NIH Postdoctoral Fellow, Department of Biochemistry, University of Colorado, Boulder \\
2020 --       & Assistant Professor, Department of Chemistry, Dickinson College, Carlisle, PA \\
\end{datetbl}

\subsection*{Honors}
\begin{datetbl}
2008            & Presidential Scholarship, Hope College, Holland, MI \\
2009            & Jaeker Chemistry Scholarship, Hope College, Holland, MI \\
2009            & Chemistry Dept. J. H. Kleinheksel Award, Hope College, Holland, MI \\
2011            & NSF International REU, Buenos Aires, Argentina \\
2011            & Barry M. Goldwater Scholarship \\
2012 -- 2017    & NSF Graduate Research Fellowship \\
2017 -- 2018    & Allergan Graduate Fellowship \\
2018            & Edward K.C. Lee Departmental Research Award \\
2019 -- 2021    & Ruth L. Kirschstein National Research Service Award, NIH Postdoctoral Fellowship \\
\end{datetbl}

%------------------------------------------------------------------------------

\section{Contribution to Science}

\begin{enumerate}

\item \textbf{Undergraduate Career:} My undergraduate contributions involved study of carbon-carbon bond activation with transition metals. C--C bond activation remains one of the final missing tools of organic synthesis. In the Johnson lab at Hope College. We studied a highly efficient C--C bond activation reaction catalyzed by a common Rhodium catalyst. Using novel kinetic techniques, I showed that C--C bond activation was the rate-determining step of catalysis, and reported the first energy of activation for this class of transformation. This work provided the field with crucial information regarding the mechanism of these reactions, and will inspire new catalytic transformations to expand the organic chemist's toolkit.

\begin{enumerate}

  \item Lutz, J. P.; \underline{Rathbun, C. M.}; Stevenson, S. M.; Powell, B. M.; Boman, T. S.; Baxter, C. E.; Zona, J. M.; Johnson, J. B. ``Rate-Limiting Step of the Rh-Catalyzed Carboacylation of Alkenes: C-C Bond Activation or Migratory Insertion?" \textit{J. Am. Chem. Soc.}, \textbf{2012}, \textit{134}, 715.

  \item \underline{Rathbun, C. M.}; Johnson, J. B. ``Rhodium-Catalyzed Acylation of Quinolinyl Ketones: Carbon-Carbon Single Bond Activation as the Turnover Limiting Step of Catalysis." \textit{J. Am. Chem. Soc.}, \textbf{2011}, \textit{133}, 2031.

\end{enumerate}


\item \textbf{Graduate Career:} In graduate school I developed new bioluminescent tools for preclinical imaging. Bioluminescent proteins are widely used for monitoring protein expression in a variety of environments. In many cases, bioluminescence is preferred over fluorescence due to its excellent sensitivity \textit{in vivo} and \textit{in vitro}. Unlike fluorescent probes, however, current bioluminescent tools lack the ability to track multiple subjects in tandem. In my graduate work, I developed the first engineered bioluminescent tools for multicomponent imaging. To find viable probes, I developed a new \textit{in silico} screening technique. Once these probes were identified, I discovered a technique to distinguish them \textit{in vivo}, and validated this method in mouse models.

\begin{enumerate}

  \item \underline{Rathbun, C. M.}; Ionkina A.; Prescher, J. A. ``Multicomponent Bioluminescence Imaging."  \textit{Manuscript in preparation.}

  \item \underline{Rathbun, C. M.}*; Porterfield, W. B.*; Jones, K. A.*; Sagoe, M. J.; Reyes, M. R.; Hua, C. T.; Prescher, J. A. ``Parallel screening for rapid identification of orthogonal bioluminescent tools." \textit{ACS Cent. Sci.}, \textbf{2017}, \textit{3}, 1254.

  % \item \underline{Rathbun, C. M.}; Prescher, J. A. ``Bioluminescent Probes for Imaging Biology Beyond the Culture Dish." \textit{Biochemistry}, \textbf{2017}, \textit{56}, 5178. \textit{Invited review}.

  \item \underline{Rathbun, C. M.}*; Jones, K. A.*; Porterfield, W. B.*; McCutcheon, D. C.; Paley, M. A.; Prescher, J. A. ``Orthogonal Luciferase–-Luciferin Pairs for Bioluminescence Imaging." \textit{J. Am. Chem. Soc.}, \textbf{2017}, \textit{139}, 2351.

  \item Steinhardt, R. C.; \underline{Rathbun, C. M.}; Krull, B. T.; Yu, J. M.; Yang Y.; Nguyen, B. D.; Kwon, J.; McCutcheon, D. C.; Jones, K. A.; Furche, F.; Prescher, J. A. ``Brominated Luciferins are Versatile Bioluminescent Probes." \textit{ChemBioChem}, \textbf{2016}, \textit{18}, 96.

\end{enumerate}

\item \textbf{Postdoctoral Career:} \comment{Some more in depth details about postdoc work.}

\begin{enumerate}

	\item \comment{Review here??}

\end{enumerate}

\end{enumerate}

\subsection*{Complete List of Published Work in MyBibliography:}
\url{https://www.ncbi.nlm.nih.gov/sites/myncbi/1lEq-HZAA5csye/bibliography/54605233/public/?sort=date&direction=ascending}


%------------------------------------------------------------------------------

\section{Additional Information: Research Support and/or Scholastic Performance}
\comment{Does this need to be changed to Research Support??}
\begin{transcript}
 & HOPE COLLEGE\centering & \\
2008 & Calculus I & TR (AP credit) \\
2008 & Calculus II & TR (AP credit) \\
2008 & Multivariable Mathematics I & A \\
2009 & Multivariable Mathematics II & A \\
2008 & General Chemistry I & A \\
2009 & General Chemistry II & A \\
%2008 & General \& Analyt Chem Lab I & A \\
%2009 & General \& Analyt Chem Lab II & A \\
2009 & General Physics I & A \\
2009 & General Physics II & A \\
%2009 & General Physics Lab I & A \\
%2009 & General Physics Lab II & A \\
2009 & Organic Chemistry I & A \\
2010 & Organic Chemistry II & A \\
%2009 & Organic Chemistry Lab I & A \\
%2010 & Organic Chemistry Lab II & A \\
2010 & Inorganic Chemistry & A \\
2010 & Inorganic Chemistry Lab & A \\
2010 & Physical Chemistry I & A \\
2011 & Physical Chemistry II & A- \\
%2010 & Physical Chemistry Lab I & A \\
%2011 & Physical Chemistry Lab II & A \\
2010 & Structure, Dynamics, and Synthesis I & A \\
2010 & Software Design \& Implementation & A \\
2011 & Data Structures \& Software Design & A \\
2011 & Statistical Methods & A- \\
2011 & Applied Statistical Models & A \\
2011 & Biochemistry I & A \\
2011 & Analytical Chemistry & A \\
%2011 & Analytical Chemistry Lab & A \\
2012 & Advanced Spectroscopy Lab & A \\
2010--2012 & Independent Research in Chemistry & A \\
 & UC IRVINE\centering & \\
2012 & Organic Reaction Mechanisms I & A \\
2012 & Organic Spectroscopy & A- \\
2012 & Organometallic Chemistry & A- \\
2013 & Organic Synthesis I & A- \\
2013 & Chemical Kinetics & A- \\
2013 & Biomacromolecules & A \\
2014 & Chemical Biology & A \\
\end{transcript}



\end{document}
